\textbf{Areitsblatt: Säuren und Basen}

\section*{Strukturformeln und Lewis-Formeln}
\begin{enumerate}
    \item Zeichne die Lewis-Formeln der folgenden Säuren:
    \begin{enumerate}
        \item Salzsäure (\ce{HCl})
        \item Schwefelsäure (\ce{H2SO4})
        \item Essigsäure (\ce{CH3COOH})
    \end{enumerate}
    \item Zeichne die Strukturformeln der folgenden Basen:
    \begin{enumerate}
        \item Ammoniak (\ce{NH3})
        \item Natriumhydroxid (\ce{NaOH})
        \item Ethylamin (\ce{C2H5NH2})
    \end{enumerate}
\end{enumerate}

\section*{Säure-Base-Reaktionen}
\begin{enumerate}
    \item Ergänze die folgenden Reaktionsgleichungen und kennzeichne die Säure-Base-Paare:
    \begin{enumerate}
        \item \ce{HCl + H2O ->}
        \item \ce{NH3 + H2O ->}
        \item \ce{H2SO4 + OH^- ->}
        \item \ce{CH3COOH + H2O ->}
    \end{enumerate}
    \item Formuliere die Reaktionsgleichung für die Neutralisation von:
    \begin{enumerate}
        \item Salzsäure mit Natriumhydroxid
        \item Schwefelsäure mit Kaliumhydroxid
        \item Essigsäure mit Ammoniak
    \end{enumerate}
\end{enumerate}

\section*{Stöchiometrische Berechnungen}
\begin{enumerate}
    \item Wie viel Gramm Natriumhydroxid (\ce{NaOH}) benötigt man, um 500 mL einer 0,1 M \ce{HCl}-Lösung vollständig zu neutralisieren? (Molare Masse von \ce{NaOH}: 40 g/mol)
    \item Wie viel Milliliter einer 0,5 M Schwefelsäure-Lösung (\ce{H2SO4}) benötigt man, um 250 mL einer 1,0 M Natronlauge (\ce{NaOH}) zu neutralisieren?
\end{enumerate}
