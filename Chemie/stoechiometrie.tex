% --- Aufgabenblatt: Stöchiometrie (10. Klasse Gymnasium) ---

\section*{Aufgabenblatt: Stöchiometrie}

\subsection*{1. Grundlagen}
\textbf{a)} Erkläre den Begriff „Stöchiometrie“ und nenne ein Beispiel aus dem Alltag.

\textbf{b)} Was versteht man unter dem Gesetz der Erhaltung der Masse?

\subsection*{2. Reaktionsgleichungen ausgleichen}
Gleiche die folgenden Reaktionsgleichungen aus:
\begin{enumerate}[a)]
    \item \ce{H2 + O2 -> H2O}
    \item \ce{Fe + O2 -> Fe2O3}
    \item \ce{C3H8 + O2 -> CO2 + H2O}
    \item \ce{Na + H2O -> NaOH + H2}
\end{enumerate}

\subsection*{3. Stoffmengen und Massen berechnen}
\textbf{a)} Wie viele Gramm Wasser entstehen bei der vollständigen Verbrennung von 10 g Wasserstoff? (Reaktionsgleichung: \ce{2H2 + O2 -> 2H2O})

\textbf{b)} Wie viele Mol Kohlenstoffdioxid entstehen bei der Verbrennung von 44 g Propan (\ce{C3H8})? (M(\ce{C3H8}) = 44 g/mol)

\textbf{c)} Wie viele Gramm Eisen(III)-oxid entstehen aus 20 g Eisen? (Reaktionsgleichung: \ce{4Fe + 3O2 -> 2Fe2O3})

\subsection*{4. Anwendungsaufgaben}
\textbf{a)} Ein Auto verbrennt 5 kg Benzin (vereinfacht: \ce{C8H18}). Wie viel kg \ce{CO2} entstehen dabei? (Reaktionsgleichung: \ce{2C8H18 + 25O2 -> 16CO2 + 18H2O})

\textbf{b)} Wie viele Liter Sauerstoff (bei Normbedingungen) werden benötigt, um 10 g Methan (\ce{CH4}) vollständig zu verbrennen? (1 mol Gas = 22{,}4 l)

\subsection*{5. Knobelaufgabe}
\textbf{a)} Bei der Photosynthese entstehen aus 6 mol \ce{CO2} und 6 mol \ce{H2O} genau 1 mol Glucose (\ce{C6H12O6}) und 6 mol \ce{O2}. Wie viele Gramm Glucose entstehen aus 132 g \ce{CO2}? (M(\ce{CO2}) = 44 g/mol, M(\ce{C6H12O6}) = 180 g/mol)

% --- Ende des Aufgabenblatts ---
