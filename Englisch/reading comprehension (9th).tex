% --- Reading Comprehension: The Last Day at Brookdale (9th Grade) ---

\section*{Reading Comprehension: \textit{"The Last Day at Brookdale"}}

\textbf{Read the following story and then answer the questions below.}

\vspace{1em}

\begin{quote}
\itshape
It was Tom’s last day at Brookdale High School. After five years, he had made lots of friends and memories. Standing at his locker, he looked around and felt a strange mixture of happiness and sadness.

He had always been shy, especially in his first year. But thanks to his friend Mia, who sat next to him in English class, he had opened up and joined the school newspaper. He became one of the editors and started writing articles about school events. That’s when he discovered his love for journalism.

Now, as he packed his books for the last time, Mia came over with a small envelope. “Don’t open it until you get home,” she said with a smile. Tom nodded, too emotional to say anything.

That evening, in his room full of boxes, he opened the envelope. Inside was a photo of the two of them at the school fair – and a note: “You’re going to be a great journalist. Never stop writing. – Mia”
\end{quote}

\vspace{1.5em}

\section*{Comprehension Questions}
\textit{Answer in full sentences.}

\begin{enumerate}
  \item How does Tom feel on his last day at Brookdale?
  \item Who helped Tom become more confident?
  \item What activity helped Tom discover his passion?
  \item What did Mia give Tom?
  \item What was written on the note in the envelope?
  \item Why do you think the photo meant a lot to Tom?
\end{enumerate}

\vspace{1.5em}

\section*{Writing Task – Letter to a Friend}
\textbf{Imagine you are Tom. Write a letter to another friend about your last day at school and your memories of Brookdale. Include the following:}

\begin{itemize}
  \item How you felt on the last day
  \item What you remember most from your time at Brookdale
  \item How your friendship with Mia helped you
  \item What your plans are for the future
\end{itemize}

\vspace{1em}

\noindent\fbox{\parbox{0.97\linewidth}{%
\textbf{Tips for your letter:}
\begin{itemize}
  \item Begin with: \textit{Dear [Name],}
  \item Write in the past tense (for school time) and future (for your plans).
  \item Write at least 120 words.
  \item End with: \textit{Best wishes,} or \textit{Take care,} and your name.
\end{itemize}
}}

\vspace{2em}

\section*{Optional Challenge}
\textit{Tom receives a reply from Mia a few months later. Write her short letter back to him. (approx. 80 words)}

% --- End of worksheet ---