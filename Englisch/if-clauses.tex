\section*{If-Clauses – Arbeitsblatt}

\textbf{Einführung:}

If-clauses (Bedingungssätze) bestehen aus zwei Teilen: dem if-Satz (Bedingung) und dem Hauptsatz (Folge). Es gibt drei Typen:

\begin{itemize}
    \item \textbf{Type I:} reale Bedingung (If + simple present, will-future)
    \item \textbf{Type II:} irreale Bedingung in der Gegenwart (If + simple past, would + infinitive)
    \item \textbf{Type III:} irreale Bedingung in der Vergangenheit (If + past perfect, would have + past participle)
\end{itemize}

\vspace{0.5em}

\textbf{1. Ergänze die Sätze (Type I):}

\begin{enumerate}[a)]
    \item If it rains, I \underline{\hspace{2cm}} (stay) at home.
    \item If you \underline{\hspace{2cm}} (study) hard, you will pass the test.
    \item If we \underline{\hspace{2cm}} (be) hungry, we will eat pizza.
    \item If my friend calls, I \underline{\hspace{2cm}} (answer) the phone.
\end{enumerate}


\vspace{0.5em}

\textbf{2. Bilde Sätze mit if (Type I):}
\begin{enumerate}[a)]
    \item you / go to bed early / you / not be tired in the morning
    \item it / snow / we / build a snowman
    \item I / have time / I / help you
\end{enumerate}

\vspace{0.5em}

\textbf{3. Ergänze die Sätze (Type II):}
% \begin{multicols}{2}
\begin{enumerate}[a)]
    \item If I \underline{\hspace{2cm}} (have) more money, I would buy a bike.
    \item If she \underline{\hspace{2cm}} (be) taller, she would play basketball.
    \item If we \underline{\hspace{2cm}} (live) in London, we would visit the Queen.
    \item If you \underline{\hspace{2cm}} (know) the answer, you would tell me.
\end{enumerate}
% \end{multicols}

\vspace{0.5em}

\textbf{4. Bilde Sätze mit if (Type II):}
\begin{enumerate}[a)]
    \item I / be you / I / travel the world
    \item we / have a car / we / drive to the beach
    \item she / speak French / she / work in Paris
\end{enumerate}

\vspace{0.5em}

\textbf{5. Ergänze die Sätze (Type III):}

\begin{enumerate}[a)]
    \item If you \underline{\hspace{2cm}} (listen) to me, you would have understood.
    \item If they \underline{\hspace{2cm}} (leave) earlier, they would have caught the bus.
    \item If I \underline{\hspace{2cm}} (see) you, I would have said hello.
    \item If we \underline{\hspace{2cm}} (know) about the party, we would have come.
\end{enumerate}


\vspace{0.5em}

\textbf{6. Bilde Sätze mit if (Type III):}
\begin{enumerate}[a)]
    \item you / study / you / pass the exam
    \item we / leave on time / we / not miss the train
    \item I / see the sign / I / stop
\end{enumerate}

\vspace{0.5em}

\textbf{7. Mische die Typen: Setze die richtige Zeitform ein (Type I, II oder III):}
\begin{enumerate}[a)]
    \item If I \underline{\hspace{2cm}} (be) you, I would call her.
    \item If it \underline{\hspace{2cm}} (rain) tomorrow, we will stay inside.
    \item If they \underline{\hspace{2cm}} (not forget) the tickets, they would have seen the show.
    \item If we \underline{\hspace{2cm}} (have) more time, we would visit you.
    \item If you \underline{\hspace{2cm}} (ask) me, I will help you.
\end{enumerate}

\vspace{0.5em}

\textbf{8. Übersetze ins Englische:}
\begin{enumerate}[a)]
    \item Wenn ich Zeit habe, helfe ich dir.
    \item Wenn sie reich wäre, würde sie ein Haus kaufen.
    \item Wenn wir das gewusst hätten, wären wir gekommen.
    \item Wenn es morgen regnet, bleiben wir zu Hause.
    \item Wenn du mich gefragt hättest, hätte ich dir geholfen.
\end{enumerate}

\vspace{0.5em}

\textbf{9. Kreativaufgabe:}

Schreibe eine kleine Geschichte (ca. 80 Wörter), in der du mindestens drei verschiedene if-clauses verwendest.

\textbf{Viel Erfolg!}