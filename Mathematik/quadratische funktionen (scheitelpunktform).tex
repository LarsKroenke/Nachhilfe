\begin{center}
    \textbf{Scheitelpunktform quadratischer Funktionen}
\end{center}

\section*{Teil A: Leichte Aufgaben (a = 1)}

Wandle die folgenden Funktionen durch quadratische Ergänzung in die Scheitelpunktform um.

\begin{enumerate}[1)]
  \item \( f(x) = x^2 + 6x + 5 \)
  \item \( f(x) = x^2 - 4x + 3 \)
  \item \( f(x) = x^2 + 2x - 8 \)
  \item \( f(x) = x^2 - 10x + 24 \)
\end{enumerate}

\vspace{1em}

\section*{Teil B: Funktionen mit \( a \ne 1 \)}

Wandle auch hier in die Scheitelpunktform um. Klammere zuerst den Faktor \( a \) aus.

\begin{enumerate}[1)]
\setcounter{enumi}{4}
  \item \( f(x) = 2x^2 + 8x + 5 \)
  \item \( f(x) = 3x^2 - 12x + 7 \)
  \item \( f(x) = -2x^2 + 4x + 1 \)
  \item \( f(x) = \frac{1}{2}x^2 - x + 4 \)
\end{enumerate}

\vspace{1em}

\section*{Teil C: Scheitelpunkt berechnen und einsetzen}

Berechne jeweils den Scheitelpunkt \((d \mid e)\) und gib die Funktion in Scheitelpunktform an.

\begin{enumerate}[1)]
\setcounter{enumi}{8}
  \item \( f(x) = 2x^2 - 4x + 3 \)
  \item \( f(x) = -x^2 + 6x - 2 \)
  \item \( f(x) = x^2 - 8x + 16 \)
  \item \( f(x) = -3x^2 - 12x - 5 \)
\end{enumerate}