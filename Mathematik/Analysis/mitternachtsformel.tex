\section*{Die Mitternachtsformel (Lösen quadratischer Gleichungen)}

\textbf{Einführung:}

Die Mitternachtsformel (auch: Lösungsformel für quadratische Gleichungen) löst Gleichungen der Form $ax^2 + bx + c = 0$:

\begin{center}
$\displaystyle x_{1,2} = \frac{-b \pm \sqrt{b^2 - 4ac}}{2a}$
\end{center}

\textbf{Begriffe:}
\begin{itemize}
    \item $a$, $b$, $c$ sind die Koeffizienten der Gleichung.
    \item $\Delta = b^2 - 4ac$ heißt Diskriminante.
    \item Ist $\Delta > 0$, gibt es zwei Lösungen; $\Delta = 0$ eine Lösung; $\Delta < 0$ keine reelle Lösung.
\end{itemize}

\vspace{0.5em}

\textbf{1. Setze ein und berechne die Diskriminante $\Delta$ (ohne zu lösen):}
\begin{multicols}{2}
\begin{enumerate}[a)]
    \item $x^2 + 4x + 3 = 0$
    \item $2x^2 - 5x + 2 = 0$
    \item $x^2 - 6x + 9 = 0$
    \item $x^2 + 2x + 5 = 0$
\end{enumerate}
\end{multicols}

\vspace{0.5em}

\textbf{2. Löse die Gleichungen mit der Mitternachtsformel:}
\begin{multicols}{2}
\begin{enumerate}[a)]
    \item $x^2 - 5x + 6 = 0$
    \item $x^2 + 2x - 8 = 0$
    \item $2x^2 - 4x - 6 = 0$
    \item $3x^2 + 6x + 3 = 0$
\end{enumerate}
\end{multicols}

\vspace{0.5em}

\textbf{3. Schwierigere Aufgaben:}
\begin{enumerate}[a)]
    \item $4x^2 - 4x + 1 = 0$
    \item $x^2 + x + 1 = 0$
    \item $5x^2 - 20x + 15 = 0$
    \item $x^2 - 7x + 10 = 0$
\end{enumerate}

\vspace{0.5em}

\textbf{4. Sachaufgabe:}

Ein Rechteck hat einen Umfang von $20$ cm und eine Fläche von $24$ cm$^2$. Stelle eine Gleichung für die Seitenlänge $x$ auf und löse sie mit der Mitternachtsformel.

\vspace{0.5em}

\textbf{5. Knobelaufgabe:}

Finde eine quadratische Gleichung mit $a = 1$, deren Lösungen $x_1 = 2$ und $x_2 = -3$ sind. Überprüfe mit der Mitternachtsformel.

\textbf{Viel Erfolg!}