\begin{center}
    \textbf{Einstieg in die Punkt- und Achsensymmetrie}
\end{center}

\section*{1. Wiederholung: Koordinatensystem}

\textbf{Aufgabe 1:} Zeichne ein Koordinatensystem mit der x-Achse von –6 bis 6 und der y-Achse von –6 bis 6.

\vspace{0.5cm}

\begin{center}
\begin{tikzpicture}
  \begin{axis}[
    axis lines=middle,
    xmin=-6, xmax=6,
    ymin=-6, ymax=6,
    grid = major,
    grid style = dotted,
    xtick={-6,-5,...,6},
    ytick={-6,-5,...,6},
    minor tick num=1,
    enlargelimits=true,
    width=10cm,
    height=10cm,
    xlabel = x,
    ylabel = y
    ]
  \end{axis}
\end{tikzpicture}
\end{center}

\textbf{Aufgabe 2:} Trage die folgenden Punkte ein: \\
A(3|2), B(–4|–1), C(0|5), D(–3|–3), E(5|0)

\vspace{0.5cm}

\section*{2. Einführung: Punktspiegelung am Ursprung}

\textbf{Merke:} Ein Punkt \( P(x|y) \) wird am Ursprung \( O(0|0) \) gespiegelt zu \( P'(–x|–y) \).

\textbf{Beispiel:} \( P(2|3) \rightarrow P'(–2|–3) \)

\vspace{0.5cm}

\textbf{Aufgabe 3:} Spiegle die folgenden Punkte am Ursprung und trage sie in dein Koordinatensystem ein: \\
a) A(3|2) \hfill b) B(–4|–1) \hfill c) C(0|5) \hfill d) D(–3|–3) \hfill e) E(5|0)

\vspace{1cm}

\section*{3. Einführung: Achsenspiegelung}

\textbf{Merke:}
\begin{itemize}
    \item Spiegelung an der \textbf{y-Achse}: \( P(x|y) \rightarrow P'(–x|y) \)
    \item Spiegelung an der \textbf{x-Achse}: \( P(x|y) \rightarrow P'(x|–y) \)
\end{itemize}

\textbf{Aufgabe 4:} Spiegle die Punkte A bis E jeweils: \\
a) an der x-Achse \\
b) an der y-Achse \\
Gib jeweils die Koordinaten der Bildpunkte an.

\vspace{1cm}

\section*{4. Zeichnerische Übungen}

\textbf{Aufgabe 5:} Zeichne ein Dreieck mit den Eckpunkten \\
A(1|2), B(4|2), C(2|5)

\vspace{0.3cm}
a) Spiegle es an der y-Achse \\
b) Spiegle es an der x-Achse \\
c) Spiegle es am Ursprung \\
Beschrifte alle Bildpunkte A', B', C' usw.

\vspace{1cm}

\section*{5. Punkt- oder Achsensymmetrie?}

\textbf{Aufgabe 6:} Entscheide, ob die folgenden Figuren achsensymmetrisch, punktsymmetrisch oder beides sind: \\
a) Quadrat \\
b) Kreis \\
c) gleichseitiges Dreieck \\
d) Parallelogramm \\
e) gleichschenkliges Trapez

\vspace{1cm}

\section*{6. Symmetrieachsen einzeichnen}

\textbf{Aufgabe 7:} Zeichne die folgenden Figuren (Gitterpunkte nutzen) und markiere mögliche Symmetrieachsen: \\
a) Quadrat \\
b) Rechteck \\
c) Kreis \\
d) Herzform \\
e) Buchstabe A

\vspace{1cm}

\section*{7. Knobelaufgaben}

\textbf{Aufgabe 8:} Der Punkt \( P(2|–3) \) wird an einem unbekannten Punkt \( M \) gespiegelt und ergibt den Bildpunkt \( P'(–1|4) \). \\
Bestimme die Koordinaten des Spiegelpunkts \( M \).

\vspace{0.5cm}

\textbf{Aufgabe 9:} Ein Dreieck hat die Punkte A(–2|1), B(1|3), C(3|0). \\
Nach einer Punktspiegelung an \( O(0|0) \) ergeben sich \( A' \), \( B' \), \( C' \). \\
Zeige rechnerisch, dass die Dreiecke ABC und \( A'B'C' \) deckungsgleich sind.

\vspace{1cm}

\section*{8. Zusatzaufgabe (für Schnelle)}

\textbf{Aufgabe 10:} Entwickle selbst eine Figur mit mindestens 6 Punkten, die achsensymmetrisch zur y-Achse ist. \\
a) Zeichne die Figur \\
b) Gib die Koordinaten der Punkte an \\
c) Begründe, warum sie achsensymmetrisch ist.