\section*{Dreiecke konstruieren – 8. Klasse}

\textbf{Wichtige Grundlagen:}

Ein Dreieck ist eindeutig konstruierbar, wenn drei Angaben (Seiten und/oder Winkel) gegeben sind. Die wichtigsten Fälle:
\begin{itemize}
    \item \textbf{SSS:} Drei Seiten
    \item \textbf{SWS:} Zwei Seiten und der eingeschlossene Winkel
    \item \textbf{WSW:} Zwei Winkel und die eingeschlossene Seite
    \item \textbf{SWW/SSW:} Zwei Seiten und ein Winkel, der nicht eingeschlossen ist
\end{itemize}

\vspace{0.5em}

\textbf{Beispiel 1: SSS-Konstruktion (Drei Seiten gegeben)}

Gegeben: $a=6\,\mathrm{cm}$, $b=4\,\mathrm{cm}$, $c=5\,\mathrm{cm}$

\textit{Schritt-für-Schritt-Anleitung:}
\begin{enumerate}
    \item Zeichne die Seite $a=6\,\mathrm{cm}$ als Grundseite.
    \item Zeichne um den linken Endpunkt einen Kreis mit Radius $b=4\,\mathrm{cm}$.
    \item Zeichne um den rechten Endpunkt einen Kreis mit Radius $c=5\,\mathrm{cm}$.
    \item Der Schnittpunkt der Kreise ist der dritte Eckpunkt. Verbinde die Punkte zu einem Dreieck.
    \item Beschrifte die Seiten und Eckpunkte.
\end{enumerate}

\begin{center}
\begin{tikzpicture}[scale=0.8]
  \draw[thick] (0,0) -- (6,0);
  \draw[dashed] (0,0) circle (4);
  \draw[dashed] (6,0) circle (5);
  \draw[thick] (0,0) -- (2.1,3.8) -- (6,0);
  \node at (0,0) [below left] {A};
  \node at (6,0) [below right] {B};
  \node at (2.1,3.8) [above] {C};
  \node at (3,-0.3) {$a$};
\end{tikzpicture}
\end{center}

\vspace{0.5em}

\textbf{Aufgabe 1:}

Konstruiere ein Dreieck mit $a=7\,\mathrm{cm}$, $b=5\,\mathrm{cm}$, $c=4\,\mathrm{cm}$ und beschrifte alle Seiten und Eckpunkte.

\vspace{0.5em}

\textbf{Beispiel 2: SWS-Konstruktion (Zwei Seiten und eingeschlossener Winkel)}

Gegeben: $b=5\,\mathrm{cm}$, $c=7\,\mathrm{cm}$, $\alpha=60^\circ$

\textit{Schritt-für-Schritt-Anleitung:}
\begin{enumerate}
    \item Zeichne die Seite $b=5\,\mathrm{cm}$.
    \item Trage am linken Endpunkt den Winkel $\alpha=60^\circ$ an.
    \item Zeichne von dort aus eine Strecke der Länge $c=7\,\mathrm{cm}$.
    \item Verbinde die Endpunkte zum Dreieck.
    \item Beschrifte die Seiten und Winkel.
\end{enumerate}

\vspace{0.5em}

\textbf{Aufgabe 2:}

Konstruiere ein Dreieck mit $b=6\,\mathrm{cm}$, $c=8\,\mathrm{cm}$, $\alpha=45^\circ$.

\vspace{0.5em}

\textbf{Beispiel 3: WSW-Konstruktion (Zwei Winkel und eingeschlossene Seite)}

Gegeben: $a=6\,\mathrm{cm}$, $\beta=40^\circ$, $\gamma=70^\circ$

\textit{Schritt-für-Schritt-Anleitung:}
\begin{enumerate}
    \item Zeichne die Seite $a=6\,\mathrm{cm}$.
    \item Trage an einem Endpunkt den Winkel $\beta=40^\circ$ an.
    \item Trage am anderen Endpunkt den Winkel $\gamma=70^\circ$ an.
    \item Die Schnittstelle der beiden Schenkel ist der dritte Eckpunkt.
    \item Verbinde die Punkte zum Dreieck und beschrifte alles.
\end{enumerate}

\vspace{0.5em}

\textbf{Aufgabe 3:}

Konstruiere ein Dreieck mit $a=5\,\mathrm{cm}$, $\beta=50^\circ$, $\gamma=60^\circ$.

\vspace{0.5em}

\textbf{4. Rechtwinkliges Dreieck}

Gegeben: $a=4\,\mathrm{cm}$, $b=3\,\mathrm{cm}$, $\gamma=90^\circ$. Zeichne das Dreieck, markiere den rechten Winkel und beschrifte die Katheten und die Hypotenuse.

\vspace{0.5em}

\textbf{5. Dreieck mit Höhe}

Gegeben: $a=6\,\mathrm{cm}$, $h_a=4\,\mathrm{cm}$, $b=5\,\mathrm{cm}$. Zeichne das Dreieck, trage die Höhe $h_a$ ein und beschrifte alle Seiten.

\vspace{0.5em}

\textbf{6. Gleichseitiges Dreieck und Höhen}

Konstruiere ein gleichseitiges Dreieck mit $a=5\,\mathrm{cm}$ und zeichne alle drei Höhen ein.

\begin{center}
\begin{tikzpicture}[scale=0.8]
  \coordinate (A) at (0,0);
  \coordinate (B) at (5,0);
  \coordinate (C) at (2.5,4.33);
  \draw[thick] (A) -- (B) -- (C) -- cycle;
  \draw[dashed] (A) -- ($(B)!0.5!(C)$);
  \draw[dashed] (B) -- ($(A)!0.5!(C)$);
  \draw[dashed] (C) -- ($(A)!0.5!(B)$);
  \node at (A) [below left] {A};
  \node at (B) [below right] {B};
  \node at (C) [above] {C};
\end{tikzpicture}
\end{center}

\vspace{0.5em}

\textbf{7. Knobelaufgabe:}

Gibt es ein Dreieck mit $a=3\,\mathrm{cm}$, $b=4\,\mathrm{cm}$, $c=8\,\mathrm{cm}$? Begründe deine Antwort mit der Dreiecksungleichung.

\vspace{0.5em}

\textbf{8. Sachaufgabe:}

Ein Grundstück hat die Form eines Dreiecks mit den Seitenlängen $a=12\,\mathrm{m}$, $b=9\,\mathrm{m}$ und $c=7\,\mathrm{m}$. Zeichne das Grundstück maßstabsgetreu ($1\,\mathrm{cm} = 1\,\mathrm{m}$) und berechne den Umfang.

\vspace{0.5em}

\textbf{9. Zusatz:}

Konstruiere ein Dreieck mit $a=6\,\mathrm{cm}$, $b=6\,\mathrm{cm}$, $\gamma=40^\circ$. Was für ein Dreieck entsteht? Begründe.

\textbf{Viel Erfolg!}