% --- Übungsblatt: Tangentengleichungen ---

\section*{Übungsblatt: Tangentengleichungen}

\subsection*{1. Grundlagen}
\textbf{a)} Erklären Sie, wie man die Gleichung der Tangente an den Graphen einer Funktion $f$ im Punkt $P(x_0|f(x_0))$ aufstellt.

\textbf{b)} Was versteht man unter der Ableitung $f'(x_0)$ an der Stelle $x_0$? Welche Bedeutung hat sie für die Tangente?

\subsection*{2. Rechenaufgaben}
\textbf{a)} Bestimmen Sie die Gleichung der Tangente an den Graphen der Funktion $f(x) = x^2$ im Punkt $P(1|f(1))$.

\textbf{b)} Bestimmen Sie die Gleichung der Tangente an den Graphen der Funktion $f(x) = 2x^3 - x$ im Punkt $P(-1|f(-1))$.

\textbf{c)} Bestimmen Sie die Gleichung der Tangente an den Graphen der Funktion $f(x) = \sin(x)$ im Punkt $P(0|f(0))$.

\textbf{d)} Bestimmen Sie die Gleichung der Tangente an den Graphen der Funktion $f(x) = e^x$ im Punkt $P(\ln(2)|f(\ln(2)))$.

\subsection*{3. Anwendungsaufgaben}
\textbf{a)} Eine Parabel ist durch $f(x) = x^2 - 4x + 3$ gegeben. Bestimmen Sie die Gleichung der Tangente im Scheitelpunkt.

\textbf{b)} Gegeben ist die Funktion $f(x) = \frac{1}{x}$. Bestimmen Sie die Gleichung der Tangente im Punkt $P(2|f(2))$.

\textbf{c)} Für welche $x$ schneidet die Tangente an den Graphen von $f(x) = x^2$ im Punkt $P(a|a^2)$ die $y$-Achse bei $y = 4$? Bestimmen Sie $a$.

\subsection*{4. Graphische Aufgaben}
\textbf{a)} Zeichnen Sie den Graphen der Funktion $f(x) = x^2$ und die Tangente im Punkt $P(2|4)$ in ein Koordinatensystem.

\textbf{b)} Zeichnen Sie den Graphen der Funktion $f(x) = \sqrt{x}$ und die Tangente im Punkt $P(1|1)$.

\subsection*{5. Zusatzaufgaben}
\textbf{a)} Eine Funktion $f$ ist an der Stelle $x_0$ nicht differenzierbar. Was bedeutet das für die Tangente in diesem Punkt?

\textbf{b)} Geben Sie ein Beispiel für eine Funktion, die an einer Stelle keine Tangente besitzt, und begründen Sie Ihre Wahl.