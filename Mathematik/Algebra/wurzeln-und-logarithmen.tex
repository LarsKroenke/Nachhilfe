\begin{center}
    \textbf{Wurzeln und Logarithmen}
\end{center}

\section*{\small Aufgabe 1: Wurzeln}
\begin{enumerate}
    \item \textbf{Berechne die folgenden Wurzeln:}
    \begin{enumerate}
        \item \(\sqrt{25}\)
        \item \(\sqrt[3]{8}\)
        \item \(\sqrt{50}\)
        \item \(\sqrt[4]{81}\)
        \item \(\sqrt{0{,}01}\)
    \end{enumerate}
    
    \item \textbf{Vereinfache die folgenden Ausdrücke:}
    \begin{enumerate}
        \item \(\sqrt{20} \cdot \sqrt{5}\)
        \item \(\sqrt{18} + \sqrt{2}\)
        \item \(\sqrt{72} - 2\sqrt{2}\)
        \item \(\frac{\sqrt{50}}{\sqrt{2}}\)
        \item \(\sqrt[3]{54} \cdot \sqrt[3]{2}\)
    \end{enumerate}
\end{enumerate}

\section*{\small Aufgabe 2: Logarithmen}
\begin{enumerate}
    \item \textbf{Berechne die folgenden Logarithmen:}
    \begin{enumerate}
        \item \(\log_{10}(1000)\)
        \item \(\log_{2}(16)\)
        \item \(\log_{5}(25)\)
        \item \(\log_{3}(1)\)
        \item \(\log_{4}(64)\)
    \end{enumerate}
    
    \item \textbf{Vereinfache die folgenden Logarithmen:}
    \begin{enumerate}
        \item \(\log_{2}(32) - \log_{2}(8)\)
        \item \(2\log_{10}(5) + \log_{10}(2)\)
        \item \(\log_{3}(9) \cdot \log_{3}(27)\)
        \item \(\frac{\log_{7}(49)}{\log_{7}(7)}\)
        \item \(\log_{a}(a^5)\)
    \end{enumerate}
\end{enumerate}

\section*{\small Aufgabe 3: Wurzeln und Logarithmen kombinieren}
\begin{enumerate}
    \item \textbf{Berechne die folgenden Ausdrücke:}
    \begin{enumerate}
        \item \(\sqrt{\log_{10}(100)}\)
        \item \(\log_{2}(\sqrt{16})\)
        \item \(\log_{5}(\sqrt[3]{125})\)
        \item \(\sqrt[3]{\log_{4}(64)}\)
        \item \(\log_{10}(\sqrt{10000})\)
    \end{enumerate}
\end{enumerate}

\section*{\small Aufgabe 4: Anwendungsaufgaben}
\begin{enumerate}
    \item \textbf{Eine radioaktive Substanz zerfällt nach einem Zeitgesetz der Form \( N(t) = N_0 \cdot e^{-kt} \), wobei \(N_0\) die Anfangsmenge und \(k\) eine Zerfallskonstante ist. Berechne die Halbwertszeit \(t_{1/2}\), d.h. die Zeit, nach der die Hälfte der Substanz zerfallen ist.}
    
    \item \textbf{Der pH-Wert einer Lösung wird durch die Gleichung \( \text{pH} = -\log_{10}[H^+] \) bestimmt, wobei \([H^+]\) die Konzentration der Wasserstoffionen in der Lösung ist. Berechne den pH-Wert einer Lösung, deren \([H^+]\) \(1 \times 10^{-4}\) beträgt.}
\end{enumerate}

\section*{\small Aufgabe 5: Zusatzaufgabe}
\begin{enumerate}
    \item \textbf{Beweise die Eigenschaft des Logarithmus:}
    \[
    \log_{a}(xy) = \log_{a}(x) + \log_{a}(y)
    \]

    \item \textbf{Zeige, dass \( \sqrt{a} \cdot \sqrt{b} = \sqrt{ab} \) für alle \( a, b \geq 0 \) gilt.}
\end{enumerate}

\section*{\small Hinweis}
Nutze für alle Aufgaben, bei denen du den Logarithmus berechnest, die Eigenschaften des Logarithmus:
\begin{itemize}
    \item \( \log_{a}(xy) = \log_{a}(x) + \log_{a}(y) \)
    \item \( \log_{a}\left(\frac{x}{y}\right) = \log_{a}(x) - \log_{a}(y) \)
    \item \( \log_{a}(x^n) = n \cdot \log_{a}(x) \)
    \item \( \log_{a}(1) = 0 \)
    \item \( \log_{a}(a) = 1 \)
\end{itemize}