\section{Was sind Logarithmen?}

Ein Logarithmus ist die Umkehrung einer Potenz. Während eine Potenz eine Zahl als Basis mit einem Exponenten multipliziert (z. B. \( 2^3 = 8 \)), gibt ein Logarithmus an, welcher Exponent benutzt wurde, um eine bestimmte Zahl zu erhalten.

Allgemein gilt:
\[
\log_b x = y \quad \Leftrightarrow \quad b^y = x
\]
Dabei ist:
\begin{itemize}
    \item \( b \) die Basis des Logarithmus (muss positiv und ungleich 1 sein),
    \item \( x \) die Zahl, für die der Logarithmus berechnet wird,
    \item \( y \) der Exponent, den man benötigt, um \( x \) aus \( b \) zu erhalten.
\end{itemize}

Beispiel:
\[
\log_2 8 = 3, \quad \text{denn } 2^3 = 8.
\]
Logarithmen haben viele Anwendungen, z. B. beim Lösen von Exponentialgleichungen, in der Physik, Chemie oder Wirtschaft.


\section{Grundlagen: Definition und Umwandlung}

\subsection*{Aufgabe 1: Umwandlung zwischen Exponential- und Logarithmenschreibweise}

\textbf{a)} Schreibe die folgenden Exponentialgleichungen in Logarithmenschreibweise um:

\begin{enumerate}[a)]
    \item \( 2^3 = 8 \)
    \item \( 10^4 = 10.000 \)
    \item \( 5^x = 25 \)
    \item \( 3^y = 81 \)
\end{enumerate}

\textbf{b)} Schreibe die folgenden Logarithmengleichungen in Exponentialschreibweise um:

\begin{enumerate}[a)]
    \item \( \log_2 16 = 4 \)
    \item \( \log_5 125 = 3 \)
    \item \( \log_3 9 = x \)
    \item \( \log_7 49 = y \)
\end{enumerate}

\section{Berechnung von Logarithmen ohne Taschenrechner}

Berechne die folgenden Logarithmen im Kopf:

\begin{enumerate}[a)]
    \item \( \log_2 8 \)
    \item \( \log_3 27 \)
    \item \( \log_4 16 \)
    \item \( \log_5 1 \)
    \item \( \log_7 49 \)
    \item \( \log_{10} 1000 \)
\end{enumerate}

\section{Anwendung der Logarithmen}

\subsection*{Aufgabe 1: Bestimmung von \( x \)}

Berechne \( x \) in den folgenden Gleichungen:

\begin{enumerate}[a)]
    \item \( \log_2 x = 5 \)
    \item \( \log_3 x = 4 \)
    \item \( \log_7 x = 2 \)
    \item \( \log_5 x = 0 \)
\end{enumerate}

\subsection*{Aufgabe 2: Lösen von Exponentialgleichungen mit Logarithmen}

Löse die folgenden Gleichungen nach \( x \) auf:

\begin{enumerate}[a)]
    \item \( 3^x = 81 \)
    \item \( 2^x = 64 \)
    \item \( 10^x = 100.000 \)
    \item \( 4^x = 32 \)
\end{enumerate}

\section{Logarithmengesetze}

Verwende die Logarithmengesetze, um die folgenden Ausdrücke zu vereinfachen:

\begin{enumerate}[a)]
    \item \( \log_2 (8 \cdot 4) \)
    \item \( \log_5 \left(\frac{25}{5}\right) \)
    \item \( \log_3 27 + \log_3 9 \)
    \item \( \log_4 16 - \log_4 2 \)
    \item \( 2 \cdot \log_3 9 \)
    \item \( \log_7 (7^x) \)
\end{enumerate}
