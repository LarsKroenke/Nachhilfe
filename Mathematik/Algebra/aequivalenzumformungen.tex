

\section*{Äquivalenzumformungen in Gleichungen}
Eine Äquivalenzumformung ist eine Umformung einer Gleichung, die ihre Lösungsmenge nicht verändert. Typische erlaubte Umformungen sind:
\begin{itemize}
    \item Addition oder Subtraktion der gleichen Zahl auf beiden Seiten
    \item Multiplikation oder Division mit der gleichen ($\neq0$) Zahl auf beiden Seiten
    \item Auflösen von Klammern und Zusammenfassen von Termen
\end{itemize}

\section*{Aufgaben}

\subsection*{1. Grundlegende Äquivalenzumformungen}
Löse die folgenden Gleichungen nach \( x \) auf:
\begin{enumerate}
    \item \( x + 5 = 12 \)
    \item \( 3x = 21 \)
    \item \( x - 8 = -3 \)
    \item \( \frac{x}{4} = 6 \)
    \item \( 5x - 2 = 3x + 8 \)
\end{enumerate}

\subsection*{2. Mehrstufige Gleichungen}
Löse die Gleichungen schrittweise und dokumentiere jede Umformung:
\begin{enumerate}
    \setcounter{enumi}{5}
    \item \( 2(x + 3) = 16 \)
    \item \( 4x - 2 = 2x + 10 \)
    \item \( 3(x - 2) + 5 = 2x + 7 \)
    \item \( 5(x + 1) - 3(x - 2) = 4x + 7 \)
    \item \( \frac{2x + 5}{3} = 7 \)
\end{enumerate}

\subsection*{3. Anwendungsaufgaben}
\begin{enumerate}
    \setcounter{enumi}{10}
    \item Ein Rechteck hat eine Länge von \( 3x + 2 \) und eine Breite von \( x + 1 \). Der Umfang beträgt 20 cm. Bestimme \( x \).
    \item Ein Vater ist dreimal so alt wie sein Sohn. In 10 Jahren wird er nur noch doppelt so alt sein. Wie alt sind beide heute?
    \item Ein Auto fährt eine Strecke mit 80 km/h und benötigt 3 Stunden. Berechne die Strecke mit einer Gleichung.
    \item Ein Schüler hat zweimal so viele Kugelschreiber wie Bleistifte. Zusammen hat er 18 Schreibgeräte. Stelle eine Gleichung auf und löse sie.
\end{enumerate}

\subsection*{4. Herausfordernde Aufgaben}
\begin{enumerate}
    \setcounter{enumi}{14}
    \item Löse die Gleichung mit Brüchen:
    \[
        \frac{2x + 1}{4} + \frac{x - 3}{2} = 5
    \]
    \item Löse die Gleichung mit Klammern:
    \[
        3(2x - 1) - 2(3x + 4) = x + 7
    \]
    \item Eine Zahl ist um 5 größer als das Vierfache einer anderen Zahl. Ihre Summe beträgt 37. Bestimme die Zahlen.
\end{enumerate}